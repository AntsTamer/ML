\documentclass[review]{elsarticle}

\usepackage{lineno,hyperref}
\usepackage{mathtools}
\usepackage{array}
\usepackage{colortbl}

\modulolinenumbers[5]


\usepackage[english,spanish]{babel}
\usepackage{blindtext}
\usepackage{xpatch}

%%%%%%%%%%%%%%%%%%%%%%%
%% Elsevier bibliography styles
%%%%%%%%%%%%%%%%%%%%%%%
%% To change the style, put a % in front of the second line of the current style and
%% remove the % from the second line of the style you would like to use.
%%%%%%%%%%%%%%%%%%%%%%%

%% Numbered
%\bibliographystyle{model1-num-names}

%% Numbered without titles
%\bibliographystyle{model1a-num-names}

%% Harvard
%\bibliographystyle{model2-names.bst}\biboptions{authoryear}

%% Vancouver numbered
%\usepackage{numcompress}\bibliographystyle{model3-num-names}

%% Vancouver name/year
%\usepackage{numcompress}\bibliographystyle{model4-names}\biboptions{authoryear}

%% APA style
%\bibliographystyle{model5-names}\biboptions{authoryear}

%% AMA style
%\usepackage{numcompress}\bibliographystyle{model6-num-names}

%% `Elsevier LaTeX' style
\bibliographystyle{elsarticle-num}
%%%%%%%%%%%%%%%%%%%%%%%
\journal{UNIR}

\begin{document}


\begin{frontmatter}

\title{Tema 3 Resumen - Introducción al método científico en investigación}

%% Group authors per affiliation:
\author{Federico Damián}




\renewcommand{\abstractname}{}    % clear the title
\begin{abstract}
	En este documento se expone un breve resumen del tema 3 de la asignatura de Investigación en Inteligencia Artificial, iniciando al lector en la investigación y sus herramientas.\cite{UNED,tema}
\end{abstract}



\begin{keyword}
Científico, \LaTeX\
\end{keyword}

\end{frontmatter}

\linenumbers

\section{Introducción}
La inteligencia artificial se trata de un ámbito
científico y esta ciencia tiene que ver con la obtención de conocimiento.\cite{WEBSITE:2}

\begin{figure}[htb]
	\centering	\includegraphics[width=50mm]{esquema.png}
	\caption{Esquema}
	\label{Esquema}
\end{figure}

\section{La ciencia y su método}
El método científico empezará por la pregunta que planteamos como marco de trabajo y continuará por los siguientes pasos:\cite{WEBSITE:1}

\begin{table}[h!]
	\begin{center}

		\begin{tabular}{l|c}
			Paso 1 & Observación\\
			Paso 2 & Hipótesis\\
			Paso 3 & Experimentación\\
			Paso 4 & Conclusión\\
			
		\end{tabular}
	\caption{Pasos del método científico.}
	\label{tab:Pasos del metodo}
	\end{center}
\end{table}.

\section{Editores de textos científicos: LaTeX}

Ejemplos de ecuaciones matemáticas:\cite{WEBSITE:3}


$$2+2=4$$

\begin{equation} \label{eq:Linearecta}
y=mx+c
\end{equation}

\begin{equation}\label{eq:Raizcuadrada}
x=\frac{-b\pm \sqrt{b^2-4ac}}{2a}
\end{equation}


\bibliography{mybibfile}

\end{document}